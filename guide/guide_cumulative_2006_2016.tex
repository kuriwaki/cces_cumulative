\documentclass[10pt,article,oneside]{memoir}
\usepackage[top=1.4in, bottom=1.4in, left=1.4in, right=1.4in]{geometry}

\usepackage{amssymb, amsmath, amsthm, mathtools}

% Essential packages
\usepackage{color}
\usepackage{graphicx}
\usepackage{tabularx}
\usepackage{longtable}
\usepackage{enumitem}
\usepackage{url}
\usepackage{float}


% Colors ------------
\usepackage{xcolor}
\definecolor{mygray}{gray}{0.5} % define color on gray scale
\definecolor{lgray}{gray}{0.9} % define color on gray scale
\definecolor{crimson}{RGB}{204,0,0}
\definecolor{dgray}{RGB}{125,125,125}
\newcommand{\crimson}[1]{\textcolor{crimson}{#1}}
  \newcommand{\dgray}[1]{\textcolor{dgray}{#1}}
    \newcommand{\lgray}[1]{\textcolor{lgray}{#1}}
      \newcommand{\mygray}[1]{\textcolor{mygray}{#1}}

% Hyperref
\usepackage[pdfencoding = auto, 
            hidelinks = true, 
            urlcolor = MidnightBlue, 
            linkcolor = MidnighBlue]{hyperref}


% Parameters
\usepackage{soul} % underline
\usepackage{tcolorbox}

% Minion for main text and math
% \usepackage{MinionPro}

% Helvetica for sans serif
% (scaled to match size of Minion)
\usepackage[scaled=0.95]{helvet}

% Bera Mono for monospaced
% (scaled to match size of Minion)
\usepackage[T1]{fontenc}
\usepackage[scaled=0.9]{beramono}
% \usepackage{fontspec}
% \setmonofont[Scale=0.85]{Monaco}

% FORCE HELVET
\renewcommand\familydefault{\sfdefault}

% Spacing
\usepackage{setspace}


% Sectioing
\setcounter{secnumdepth}{0} % fix starting with "0.1"



% Theorems
\usepackage{etoolbox}
\renewcommand{\qedsymbol}{\rule{0.7em}{0.7em}}
\newtheorem{theorem}{Theorem}
\newtheorem{lem}{Lemma}
\newtheorem{assump}{Assumption}
\theoremstyle{definition}
\newtheorem{definition}{Definition}
\AtEndEnvironment{definition}{\null\hfill\qedsymbol}%
\newtheorem{example}{Example}
\newtheorem{error}{Common Error}
\AtEndEnvironment{error}{\null\hfill\qedsymbol}%

% Math
\newcommand{\var}{\textnormal{Var}}
\newcommand{\cov}{\textnormal{Cov}} 
\newcommand{\cor}{\textnormal{Cor}} 

% Figures
\usepackage{graphicx,grffile}
\makeatletter
\def\maxwidth{\ifdim\Gin@nat@width>\linewidth\linewidth\else\Gin@nat@width\fi}
\def\maxheight{\ifdim\Gin@nat@height>\textheight\textheight\else\Gin@nat@height\fi}
\makeatother
% Scale images if necessary, so that they will not overflow the page
% margins by default, and it is still possible to overwrite the defaults
% using explicit options in \includegraphics[width, height, ...]{}
\setkeys{Gin}{width=\maxwidth,height=\maxheight,keepaspectratio}


% Shading and Rmd macros
\usepackage{color}
\usepackage{fancyvrb}
\newcommand{\VerbBar}{|}
\newcommand{\VERB}{\Verb[commandchars=\\\{\}]}
\DefineVerbatimEnvironment{Highlighting}{Verbatim}{commandchars=\\\{\}}
% Add ',fontsize=\small' for more characters per line
\usepackage{framed}
\definecolor{shadecolor}{RGB}{248,248,248}
\newenvironment{Shaded}{\begin{snugshade}}{\end{snugshade}}
\newcommand{\KeywordTok}[1]{\textcolor[rgb]{0.13,0.29,0.53}{\textbf{#1}}}
\newcommand{\DataTypeTok}[1]{\textcolor[rgb]{0.13,0.29,0.53}{#1}}
\newcommand{\DecValTok}[1]{\textcolor[rgb]{0.00,0.00,0.81}{#1}}
\newcommand{\BaseNTok}[1]{\textcolor[rgb]{0.00,0.00,0.81}{#1}}
\newcommand{\FloatTok}[1]{\textcolor[rgb]{0.00,0.00,0.81}{#1}}
\newcommand{\ConstantTok}[1]{\textcolor[rgb]{0.00,0.00,0.00}{#1}}
\newcommand{\CharTok}[1]{\textcolor[rgb]{0.31,0.60,0.02}{#1}}
\newcommand{\SpecialCharTok}[1]{\textcolor[rgb]{0.00,0.00,0.00}{#1}}
\newcommand{\StringTok}[1]{\textcolor[rgb]{0.31,0.60,0.02}{#1}}
\newcommand{\VerbatimStringTok}[1]{\textcolor[rgb]{0.31,0.60,0.02}{#1}}
\newcommand{\SpecialStringTok}[1]{\textcolor[rgb]{0.31,0.60,0.02}{#1}}
\newcommand{\ImportTok}[1]{#1}
\newcommand{\CommentTok}[1]{\textcolor[rgb]{0.56,0.35,0.01}{\textit{#1}}}
\newcommand{\DocumentationTok}[1]{\textcolor[rgb]{0.56,0.35,0.01}{\textbf{\textit{#1}}}}
\newcommand{\AnnotationTok}[1]{\textcolor[rgb]{0.56,0.35,0.01}{\textbf{\textit{#1}}}}
\newcommand{\CommentVarTok}[1]{\textcolor[rgb]{0.56,0.35,0.01}{\textbf{\textit{#1}}}}
\newcommand{\OtherTok}[1]{\textcolor[rgb]{0.56,0.35,0.01}{#1}}
\newcommand{\FunctionTok}[1]{\textcolor[rgb]{0.00,0.00,0.00}{#1}}
\newcommand{\VariableTok}[1]{\textcolor[rgb]{0.00,0.00,0.00}{#1}}
\newcommand{\ControlFlowTok}[1]{\textcolor[rgb]{0.13,0.29,0.53}{\textbf{#1}}}
\newcommand{\OperatorTok}[1]{\textcolor[rgb]{0.81,0.36,0.00}{\textbf{#1}}}
\newcommand{\BuiltInTok}[1]{#1}
\newcommand{\ExtensionTok}[1]{#1}
\newcommand{\PreprocessorTok}[1]{\textcolor[rgb]{0.56,0.35,0.01}{\textit{#1}}}
\newcommand{\AttributeTok}[1]{\textcolor[rgb]{0.77,0.63,0.00}{#1}}
\newcommand{\RegionMarkerTok}[1]{#1}
\newcommand{\InformationTok}[1]{\textcolor[rgb]{0.56,0.35,0.01}{\textbf{\textit{#1}}}}
\newcommand{\WarningTok}[1]{\textcolor[rgb]{0.56,0.35,0.01}{\textbf{\textit{#1}}}}
\newcommand{\AlertTok}[1]{\textcolor[rgb]{0.94,0.16,0.16}{#1}}
\newcommand{\ErrorTok}[1]{\textcolor[rgb]{0.64,0.00,0.00}{\textbf{#1}}}
\newcommand{\NormalTok}[1]{#1}


% Parmaters

\title{ \LARGE\textbf{CCES Cumulative Common Content (2006 - 2016)}}

\author{Shiro Kuriwaki\thanks{Department of Government, Harvard University. Email:
\url{kuriwaki@g.harvard.edu}. My thanks to Alexander Agadjanian, Steve
Ansolabehere, Stephen DiMauro, Nathan Kaplan, Stephen Pettigrew, Joe
Williams, and the Crunch Team for their contributions and guidance.}  }


\date{Guide last updated: 2018-02-17}

\begin{document}

\maketitle





\renewcommand\UrlFont{\color{crimson}\ttfamily}
















\begin{quote}
Kuriwaki, Shiro, 2018, ``Cumulative CCES Common Content (2006-2016)'',
\href{https://dataverse.harvard.edu/dataset.xhtml?persistentId=doi:10.7910/DVN/II2DB6}{\url{doi:10.7910/DVN/II2DB6}},
Harvard Dataverse
\end{quote}

\noindent This dataset combines eleven years (2006 - 2016) of the
Cooperative Congressional Election Study (Principal Investigators:
Stephen Ansolabehere, Brian Schaffner, and Sam Luks).

The Cooperative Congressional Election Study (CCES) is an online survey
conducted around November of each year, asking a range of questions on
political behavior and public opinion. While questions can change from
year to year, some standard questions like the ones this cumulative file
includes allow for interesting comparisons.

This dataset was constructed based off CCES datasets from each year. A
set of R scripts formatted, merged, and standardized these datasets to
generate a \texttt{tibble}-style data frame. In addition, the same
dataset is available on \texttt{Crunch}, an accessible analytics
interface optimized for survey datasets.

Please note that this cumulative dataset makes modifications to the
original CCES datasets for comparability. These modifications are only
made when differences are deemed sufficiently minor, and are documented
in source code (see below). However, for details on the survey
methodology and a list of all questions, readers should consult the
guides for each year.

\bigskip

\begin{tcolorbox}[boxrule=0pt,  fonttitle=\bfseries, subtitle style={boxrule=0.4pt, colback=black!75!white}]

\tcbsubtitle{To see the source code, }
report a bug, or ask a question about the data, please feel free to file an issue from the source code page:  \url{https://github.com/kuriwaki/cces_cumulative}. Alternatively, please contact me by email.

\tcbsubtitle{To obtain the individual year's CCES datasets,}
search the CCES dataverse (\url{https://dataverse.harvard.edu/dataverse/cces}) or access the CCES homepage (\url{https://cces.gov.harvard.edu/}).

\tcbsubtitle{To examine the survey methodology, }
consult the Methodology section of the most recent Common Content's codebook: \url{https://dataverse.harvard.edu/dataset.xhtml?persistentId=doi:10.7910/DVN/GDF6Z0}.

\end{tcolorbox}

\bigskip

\newpage

\section{Getting Started}\label{getting-started}

The \texttt{.Rds} format can be read into R. This format preserves
dataset properties such as the distinction between integers and doubles,
and labelled variables.

\begin{Shaded}
\begin{Highlighting}[]
\NormalTok{df <-}\StringTok{ }\KeywordTok{readRDS}\NormalTok{(}\StringTok{"cumulative_2006_2016.Rds"}\NormalTok{)}
\end{Highlighting}
\end{Shaded}

The dataset in R is best viewed in \texttt{dplyr}, although it can be
analyzed as a standard data frame.

\begin{Shaded}
\begin{Highlighting}[]
\KeywordTok{library}\NormalTok{(tidyverse)}
\NormalTok{df}
\end{Highlighting}
\end{Shaded}

\begin{verbatim}
# A tibble: 374,556 x 68
    year case_id weight weight_cumulative state  st    cd     dist dist_up
   <int>   <int>  <dbl>             <dbl> <chr>  <chr> <chr> <int>   <int>
 1  2006  439219  1.85              1.67  North~ NC    NC-10    10      10
 2  2006  439224  0.968             0.872 Ohio   OH    OH-3      3       3
 3  2006  439228  1.59              1.44  New J~ NJ    NJ-1      1       1
 4  2006  439237  1.40              1.26  Illin~ IL    IL-9      9       9
 5  2006  439238  0.903             0.813 New Y~ NY    NY-22    22      22
 6  2006  439242  0.839             0.756 Texas  TX    TX-11    11      11
 7  2006  439251  0.777             0.700 Minne~ MN    MN-3      3       3
 8  2006  439254  0.839             0.756 Nevada NV    NV-2      2       2
 9  2006  439255  0.331             0.299 Texas  TX    TX-24    24      24
10  2006  439263  1.10              0.993 Maryl~ MD    MD-2      2       2
# ... with 374,546 more rows, and 59 more variables: cong <int>,
#   cong_up <int>, zipcode <chr>, county_fips <chr>, tookpost <int+lbl>,
#   weight_post <dbl>, starttime <dttm>, pid3 <int+lbl>, pid7 <int+lbl>,
#   pid3_leaner <int+lbl>, gender <int+lbl>, birthyr <int>, age <int>,
#   race <int+lbl>, hispanic <int+lbl>, educ <int+lbl>,
#   economy_retro <int+lbl>, approval_pres <int+lbl>, approval_rep <fct>,
#   approval_sen1 <fct>, approval_sen2 <fct>, approval_gov <int+lbl>,
#   intent_pres_08 <fct>, intent_pres_12 <fct>, intent_pres_16 <fct>,
#   voted_pres_08 <fct>, voted_pres_12 <fct>, voted_pres_16 <fct>,
#   vv_regstatus <fct>, vv_party_gen <fct>, vv_party_prm <fct>,
#   vv_turnout_gvm <fct>, vv_turnout_pvm <fct>, intent_rep <fct>,
#   intent_sen <fct>, intent_gov <fct>, voted_rep <fct>, voted_sen <fct>,
#   voted_gov <fct>, intent_rep_chosen <chr>, intent_rep_fec <chr>,
#   intent_sen_chosen <chr>, intent_sen_fec <chr>,
#   intent_gov_chosen <chr>, intent_gov_fec <chr>, voted_rep_chosen <chr>,
#   voted_rep_fec <chr>, voted_sen_chosen <chr>, voted_sen_fec <chr>,
#   voted_gov_chosen <chr>, voted_gov_fec <chr>, rep_current <chr>,
#   rep_icpsr <int>, sen1_current <chr>, sen1_icpsr <int>,
#   sen2_current <chr>, sen2_icpsr <int>, gov_current <chr>, gov_fec <chr>
\end{verbatim}

A Stata dta file is provided as well.
\texttt{cumulative\_2006\_2016.dta} can be read by Stata, or in R by the
\texttt{haven} package

\begin{Shaded}
\begin{Highlighting}[]
\KeywordTok{library}\NormalTok{(haven)}
\NormalTok{df <-}\StringTok{ }\KeywordTok{read_dta}\NormalTok{(}\StringTok{"cumulative_2006_2016.dta"}\NormalTok{)}
\end{Highlighting}
\end{Shaded}

\section{Features of the 2006 - 2016 Cumulative
Dataset}\label{features-of-the-2006---2016-cumulative-dataset}

\subsection{Unified Variable Names}\label{unified-variable-names}

Most variables in this dataset come straight from each year's CCES.
However, it renames and standardizes variable names, making them
accessible in one place. Please see the rest of this guide or the Crunch
dataset for a full list and description of variables.

\subsection{Chosen Candidate Names and
Identifiers}\label{chosen-candidate-names-and-identifiers}

One addition to this cumulative dataset are variables of candidate names
and identifiers that a respondent chose. In the individual year's CCES
datasets, typically the response values for a vote choice question is a
generic label, e.g. \texttt{Candidate1} and \texttt{Candidate2}. Then,
separate variables of names and parties correspond to each
\texttt{Candidate1} and \texttt{Candidate2}.

Instead, the cumulative dataset shows both the generic label \emph{and}
the chosen candidate's name, party, and identifier, which will vary
across individuals.

\begin{Shaded}
\begin{Highlighting}[]
\KeywordTok{select}\NormalTok{(df, year, case_id, st, }\KeywordTok{matches}\NormalTok{(}\StringTok{"voted_sen"}\NormalTok{))}
\end{Highlighting}
\end{Shaded}

\begin{verbatim}
# A tibble: 374,556 x 6
    year case_id st    voted_sen         voted_sen_chosen    voted_sen_fec
   <int>   <int> <chr> <fct>             <chr>               <chr>        
 1  2006  439219 NC    <NA>              <NA>                <NA>         
 2  2006  439224 OH    [Democrat / Cand~ Sherrod C. Brown (~ S6OH00163    
 3  2006  439228 NJ    [Democrat / Cand~ Robert Menendez (D) S6NJ00289    
 4  2006  439237 IL    <NA>              <NA>                <NA>         
 5  2006  439238 NY    [Democrat / Cand~ Hillary Rodham Cli~ S0NY00188    
 6  2006  439242 TX    I Did Not Vote I~ <NA>                <NA>         
 7  2006  439251 MN    [Republican / Ca~ Mark Kennedy (R)    S6MN00275    
 8  2006  439254 NV    [Democrat / Cand~ Jack Carter (D)     S6NV00150    
 9  2006  439255 TX    [Democrat / Cand~ Barbara Ann Radnof~ S6TX00180    
10  2006  439263 MD    I Did Not Vote I~ <NA>                <NA>         
# ... with 374,546 more rows
\end{verbatim}

\subsection{Crunch}\label{crunch}

A version of the dataset is also included in Crunch, a database platform
that makes it easy to view and analyze survey data either with our
without any programming experience. Crunch is in beta at the time of
writing.

\begin{enumerate}
\def\labelenumi{\arabic{enumi}.}
\tightlist
\item
  Obtain Access: For View access to the dataset (free), please sign up
  here:
  \url{https://harvard.az1.qualtrics.com/jfe/form/SV_066hQi4Eeco3Kap}.
  For questions and more access, please contact the CCES Team.
\end{enumerate}

\newpage

\begin{enumerate}
\def\labelenumi{\arabic{enumi}.}
\setcounter{enumi}{1}
\tightlist
\item
  Browse: Crunch offers a web GUI for quickly browsing variables:
\end{enumerate}

\begin{figure}[H]
\centering
\centerline{\includegraphics[width=1.05\linewidth]{01_crunch_browse.png}}
\end{figure}

\begin{enumerate}
\def\labelenumi{\arabic{enumi}.}
\setcounter{enumi}{2}
\tightlist
\item
  Analyze: The crunch interface allows Viewers to make cross-tabs and
  bar graphs quickly.\\

  \begin{figure}[H]
  \centering
  \centerline{\includegraphics[width=1.05\linewidth]{02_crunch_tab.png}}
  \end{figure}
\end{enumerate}

Crunch datasets can also be manipulated from a R package,
\texttt{crunch} \url{https://github.com/Crunch-io/rcrunch}.

\newpage

\section{Variables}\label{variables}

The sections below provide summary more information on each variable.

\begin{itemize}
\tightlist
\item
  The title shows the name as used in the dataset, suitable for coding
  (``alias'' in Crunch terminology). followed by a more descriptive.
  name suitable for presentation (``name'' in Crunch terminology).
\item
  Question wordings, where applicable, immediately follow. Otherwise an
  description is provide in square brackets (\texttt{{[}\ \ {]}}). All
  square brackets, both in the description and the response options,
  indicate descriptions that are summaries of what the respondent saw
  rather than the question verbatim.
\item
  A tabulation of response options (or summary statistics for numeric
  variables) follow. Numbers are unweighted counts.
\item
  The ``Years'' bullet lists the years of the CCES in which data on the
  variable is available at all. If a year is not listed, either the
  question was not asked in the year or was not incorporated in the
  creation of this dataset.
\item
  Finally, the ``Limitations'' bullet notes some of the caveats required
  when interpreting this variable. As this dataset is combinations of
  different surveys, some year-specific details on implementation are
  inevitably lost. For example, for all 2016 responses ``Not Asked'' and
  ``Skipped'' are both coded as a \texttt{NA} (missing) to stay
  consistent with past years that did not make that finer distinction.
\end{itemize}

\section{Administration}\label{administration}

\subsubsection{\texorpdfstring{\texttt{year}: CCES
year}{year: CCES year}}\label{year-cces-year}

{[}Year of CCES Common Content{]}

\begin{verbatim}
 year     n
 2006 36421
 2007 10000
 2008 32800
 2009 13800
 2010 55400
 2011 20150
 2012 54535
 2013 16400
 2014 56200
 2015 14250
 2016 64600
\end{verbatim}

\subsubsection{\texorpdfstring{\texttt{starttime}: Start
time}{starttime: Start time}}\label{starttime-start-time}

{[}Pre-election wave start time (up to second){]}

\begin{verbatim}
                 Min.               1st Qu.                Median 
"2006-10-07 00:02:34" "2010-10-01 19:36:34" "2012-10-07 10:30:42" 
                 Mean               3rd Qu.                  Max. 
"2012-06-06 07:11:46" "2014-10-18 06:01:53" "2016-11-07 16:46:42" 
\end{verbatim}

\begin{itemize}
\tightlist
\item
  Years: All of 2006-2016
\end{itemize}

\subsubsection{\texorpdfstring{\texttt{tookpost}: Took post-election
wave}{tookpost: Took post-election wave}}\label{tookpost-took-post-election-wave}

{[}Whether or not the respondent took the post-election wave of the
survey (in even years){]}

\begin{verbatim}
                          tookpost      n
 Did Not Take Post-Election Survey  50872
         Took Post-Election Survey 249084
                              <NA>  74600
\end{verbatim}

\begin{itemize}
\tightlist
\item
  Years: 2006, 2008, 2010, 2012, 2014, 2016 (Post-election wave only
  exists for even years)
\end{itemize}

\section{Weights}\label{weights}

\subsubsection{\texorpdfstring{\texttt{weight}: Survey weight
(Year-Specific)}{weight: Survey weight (Year-Specific)}}\label{weight-survey-weight-year-specific}

{[}weights from pre-election survey of each year{]}

\begin{verbatim}
   Min. 1st Qu.  Median    Mean 3rd Qu.    Max. 
 0.0000  0.4094  0.7226  1.0000  1.1849 15.0006 
\end{verbatim}

\begin{itemize}
\tightlist
\item
  Years: All of 2006-2016
\item
  In even years, they are re-computed after vote validation has been
  computed and those re-computed weights are taken here when available.
  The weights applied to the sample (which is originally drawn from a
  matched sample) are constructed to make each year's respondents' pool
  representative of the national adult population. See the methodology
  section of the
  \href{https://dataverse.harvard.edu/api/access/datafile/3047286}{2016
  Guide} for details.
\item
  Limitations: Only specific to each year. Built off of the entire
  pre-election wave sample, but not necessarily to adjust post-election
  wave respondents. See \texttt{weight\_post}
\end{itemize}

\subsubsection{\texorpdfstring{\texttt{weight\_cumulative}: Survey
weight
(Cumulative)}{weight\_cumulative: Survey weight (Cumulative)}}\label{weight_cumulative-survey-weight-cumulative}

{[}weight variable with simple adjustment: multiplied a constant within
year to make years comparable{]}

\begin{verbatim}
   Min. 1st Qu.  Median    Mean 3rd Qu.    Max. 
 0.0000  0.2985  0.5839  0.9633  1.1182 24.0297 
\end{verbatim}

\begin{itemize}
\tightlist
\item
  Years: All of 2006-2016
\item
  Limitations: Only a simple transformation of \texttt{weight}
\end{itemize}

\subsubsection{\texorpdfstring{\texttt{weight\_post}: Survey weight for
post-election
wave}{weight\_post: Survey weight for post-election wave}}\label{weight_post-survey-weight-for-post-election-wave}

{[}weight for post-election wave respondents. Only available for some of
the even years.{]}

\begin{verbatim}
   Min. 1st Qu.  Median    Mean 3rd Qu.    Max.    NA's 
   0.00    0.43    0.71    1.00    1.13   15.00  276779 
\end{verbatim}

\begin{itemize}
\tightlist
\item
  Years: 2012, 2016
\item
  Limitations: Only available for some even years.
\end{itemize}

\section{Geography}\label{geography}

A series of variables for the respondent's location

\begin{itemize}
\tightlist
\item
  \texttt{state}: State: {[}State (Imputed from input zipcode){]}
\item
  \texttt{st}: State abbreviation: {[}State (Imputed from input
  zipcode){]}
\item
  \texttt{dist}: Congressional district number in current Congress:
  {[}Current Congressional District Number (Imputed from input
  zipcode){]}
\item
  \texttt{dist\_up}: Congressional district number for upcoming
  Congress: {[}Upcoming Congressional District Number (Imputed from
  input zipcode){]}
\item
  \texttt{cd}: Congressional district in current Congress: {[}Current
  Congressional District (Imputed from input zipcode){]}
\item
  \texttt{zipcode}: Zipcode of residence: So that we can ask you about
  the news and events in your area, in what zip code do you currently
  reside?
\item
  \texttt{county\_fips}: County of residence: {[}County (Imputed from
  input zipcode){]}
\end{itemize}

\begin{verbatim}
Observations: 374,556
Variables: 7
$ state       <chr> "California", "Pennsylvania", "Texas", "Texas", "T...
$ st          <chr> "CA", "PA", "TX", "TX", "TX", "NY", "NC", "NC", "M...
$ cd          <chr> "CA-2", "PA-5", "TX-16", "TX-19", "TX-6", "NY-28",...
$ dist        <int> 2, 5, 16, 19, 6, 28, 11, 7, 1, 17, 15, 1, 2, 6, 1,...
$ dist_up     <int> 1, 3, 16, 19, 6, 27, 11, 7, 2, 20, 12, 1, 2, 8, 1,...
$ zipcode     <chr> "95969", "16255", "79924", "79423", "76123", "1413...
$ county_fips <chr> "06007", "42031", "48141", "48303", "48439", "3606...
\end{verbatim}

\begin{itemize}
\tightlist
\item
  Years: All of 2006-2016
\item
  Limitations: Some years do not provide the variable relevant to
  \texttt{dist\_up}, in which case the current district (\texttt{dist})
  is assigned automatically. Thus, \texttt{dist\_up} may not reflect,
  for example, district changes in off-cycle redistricting. Only
  residence (not registration) geographies included here; see individual
  years' for registration geographies.
\end{itemize}

\section{Demographics}\label{demographics}

\subsubsection{\texorpdfstring{\texttt{gender}:
Gender}{gender: Gender}}\label{gender-gender}

Are you male or female?

\begin{verbatim}
 gender      n
   Male 176528
 Female 198028
\end{verbatim}

\begin{itemize}
\tightlist
\item
  Years: All of 2006-2016
\end{itemize}

\subsubsection{\texorpdfstring{\texttt{birthyr}: Year of
birth}{birthyr: Year of birth}}\label{birthyr-year-of-birth}

In what year were you born?

\begin{verbatim}
   Min. 1st Qu.  Median    Mean 3rd Qu.    Max. 
   1900    1949    1960    1962    1975    1998 
\end{verbatim}

\begin{itemize}
\tightlist
\item
  Years: All of 2006-2016
\end{itemize}

\subsubsection{\texorpdfstring{\texttt{age}:
Age}{age: Age}}\label{age-age}

{[}Approximate age computed from the year of survey minus Year of
Birth{]}

\begin{verbatim}
   Min. 1st Qu.  Median    Mean 3rd Qu.    Max. 
  18.00   37.00   52.00   49.92   62.00  109.00 
\end{verbatim}

\begin{itemize}
\tightlist
\item
  Years: All of 2006-2016
\end{itemize}

\subsubsection{\texorpdfstring{\texttt{educ}:
Education}{educ: Education}}\label{educ-education}

What is the highest level of education you have completed?

\begin{verbatim}
                 educ      n
                No HS  11128
 High School Graduate 103364
         Some College  95472
               2-Year  35368
               4-Year  85044
            Post-Grad  44113
                 <NA>     67
\end{verbatim}

\begin{itemize}
\tightlist
\item
  Years: All of 2006-2016
\end{itemize}

\subsubsection{\texorpdfstring{\texttt{race}:
Race}{race: Race}}\label{race-race}

What racial or ethnic group best describes you?

\begin{verbatim}
            race      n
           White 280670
           Black  41334
        Hispanic  28449
           Asian   6991
 Native American   2937
           Mixed   6901
           Other   6707
  Middle Eastern    567
\end{verbatim}

\begin{itemize}
\tightlist
\item
  Years: All of 2006-2016
\item
  Limitations: The ``Hispanic'' value may undercount self-identified
  Hispanics. See \texttt{hispanic}
\end{itemize}

\subsubsection{\texorpdfstring{\texttt{hispanic}:
Hispanic}{hispanic: Hispanic}}\label{hispanic-hispanic}

Are you of Spanish, Latino, or Hispanic origin or descent? {[}Asked if
response to race is not Hispanic{]}

\begin{verbatim}
 hispanic      n
      Yes   8356
       No 249115
     <NA> 117085
\end{verbatim}

\begin{itemize}
\tightlist
\item
  Years: 2010, 2011, 2012, 2013, 2014, 2015, 2016
\item
  In years in which this question was fielded, this question supplements
  the \texttt{race} variable by asking does who did \emph{not} respnd
  ``Hipsanic'' in the \texttt{race} question.
\end{itemize}

\newpage

\section{Validations}\label{validations}

\subsubsection{\texorpdfstring{\texttt{vv\_regstatus}: Validated
registration
status}{vv\_regstatus: Validated registration status}}\label{vv_regstatus-validated-registration-status}

{[}Validation results. Missing if validation was not conducted in the
year. Categories are aggregated. Both Matched-not registered and
unmatched are labeled as a no record.{]}

\begin{verbatim}
              vv_regstatus      n
                    Active 178356
 No Record Of Registration  61861
              Unregistered  13826
                   Dropped   5294
                  Inactive   3047
      Multiple Appearances   1151
                      <NA> 111021
\end{verbatim}

\begin{itemize}
\tightlist
\item
  Years: 2008, 2010, 2012, 2014, 2016
\item
  Limitations: Collapses some response options
\end{itemize}

\subsubsection{\texorpdfstring{\texttt{vv\_party\_gen}: Validated
registered
party}{vv\_party\_gen: Validated registered party}}\label{vv_party_gen-validated-registered-party}

{[}Validation results{]}

\begin{verbatim}
                    vv_party_gen      n
 No Record Of Party Registration  60890
                         Unknown  51379
                Democratic Party  27058
                Republican Party  21522
            No Party Affiliation   9835
               Declined To State   1579
                           Other   1286
               Independent Party   1176
              Liberatarian Party    376
                     Green Party    194
              Constitution Party     27
                    Reform Party      9
                 Socialist Party      3
                             Cns      1
                            <NA> 199221
\end{verbatim}

\begin{itemize}
\tightlist
\item
  Years: 2012, 2014, 2016
\item
  Limitations: Not available for some even years
\end{itemize}

\subsubsection{\texorpdfstring{\texttt{vv\_party\_prm}: Validated
registered Primary
party}{vv\_party\_prm: Validated registered Primary party}}\label{vv_party_prm-validated-registered-primary-party}

{[}Validation results. All vote methods (polling, mail, early, unknown,
etc..) are aggregated as a vote.{]}

\begin{verbatim}
                    vv_party_prm      n
 No Record Of Party Registration 157120
                Republican Party  10010
                Democratic Party   8202
                           Other      3
                            <NA> 199221
\end{verbatim}

\begin{itemize}
\tightlist
\item
  Years: 2012, 2014, 2016
\item
  Limitations: Not available for some even years
\end{itemize}

\subsection{Turnout}\label{turnout}

\subsubsection{\texorpdfstring{\texttt{vv\_turnout\_gvm}: Validated
turnout General
Election}{vv\_turnout\_gvm: Validated turnout General Election}}\label{vv_turnout_gvm-validated-turnout-general-election}

{[}Validation results. All vote methods (polling, mail, early, unknown,
etc..) are aggregated as a vote.{]}

\begin{verbatim}
      vv_turnout_gvm      n
               Voted 169204
 No Record Of Voting 129019
       No Voter File   1733
                <NA>  74600
\end{verbatim}

\begin{itemize}
\tightlist
\item
  Years: 2006, 2008, 2010, 2012, 2014, 2016
\item
  Limitations: Collapses most response options. In particular, the
  particular voting method is collapsed into one category, even though
  \texttt{gvm} stands for General Election voting method. Also, the
  result of not matching to a voter file is collapsed with the result of
  matching to a voter file and having no indication of turning out to
  vote. The distinction is unclear in earlier years, and is thus
  collapsed for all years here. For finer distinctions, see the
  individual year's CCES.
\end{itemize}

\subsubsection{\texorpdfstring{\texttt{vv\_turnout\_pvm}: Validated
turnout Primary Election
(Congressional)}{vv\_turnout\_pvm: Validated turnout Primary Election (Congressional)}}\label{vv_turnout_pvm-validated-turnout-primary-election-congressional}

{[}Validation results{]}

\begin{verbatim}
      vv_turnout_pvm      n
 No Record Of Voting 185927
               Voted  76245
       No Voter File   1363
                <NA> 111021
\end{verbatim}

\begin{itemize}
\tightlist
\item
  Years: 2008, 2010, 2012, 2014, 2016
\item
  Limitations: See \texttt{vv\_turnout\_gvm}
\end{itemize}

\newpage

\section{Identity and Attitudes}\label{identity-and-attitudes}

\subsection{Partisan Identity}\label{partisan-identity}

\subsubsection{\texorpdfstring{\texttt{pid3}: Partisan identity (3
point)}{pid3: Partisan identity (3 point)}}\label{pid3-partisan-identity-3-point}

Generally speaking, do you think of yourself as a \ldots{}?

\begin{verbatim}
        pid3      n
    Democrat 132407
  Republican  98672
 Independent 104296
       Other  14597
    Not Sure  15684
        <NA>   8900
\end{verbatim}

\begin{itemize}
\tightlist
\item
  Years: All of 2006-2016
\item
  Limitations: Response options offer slightly by year. For example, the
  \texttt{Not\ Sure} option is not a response option in years 2006 and
  2010. Open-text responses not included. 2010 values are from the
  post-election wave.
\end{itemize}

\subsubsection{\texorpdfstring{\texttt{pid7}: Partisan identity (7
point)}{pid7: Partisan identity (7 point)}}\label{pid7-partisan-identity-7-point}

{[}Based on branching from Partisan Identity question{]}

\begin{verbatim}
                       pid7     n
            Strong Democrat 89117
   Not Very Strong Democrat 45242
              Lean Democrat 37432
                Independent 49221
            Lean Republican 40464
 Not Very Strong Republican 36225
          Strong Republican 63147
                   Not Sure 10783
                       <NA>  2925
\end{verbatim}

\begin{itemize}
\tightlist
\item
  Years: All of 2006-2016
\item
  Limitations: See \texttt{pid3}
\end{itemize}

\subsubsection{\texorpdfstring{\texttt{pid3\_leaner}: Partisan identity
(including
leaners)}{pid3\_leaner: Partisan identity (including leaners)}}\label{pid3_leaner-partisan-identity-including-leaners}

{[}Codes self-identified Independents in pid3 who expressed leaning
towards a party in pid7 (Lean Democrats / Republicans) as partisans.{]}

\begin{verbatim}
                     pid3_leaner      n
    Democrat (Including Leaners) 171791
  Republican (Including Leaners) 139836
 Independent (Excluding Leaners)  49221
                        Not Sure  10783
                            <NA>   2925
\end{verbatim}

\begin{itemize}
\tightlist
\item
  Years: All of 2006-2016
\item
  Limitations: See \texttt{pid3}
\end{itemize}

\subsection{Economy}\label{economy}

\subsubsection{\texorpdfstring{\texttt{economy\_retro}: Retrospective
economy}{economy\_retro: Retrospective economy}}\label{economy_retro-retrospective-economy}

OVER THE PAST YEAR the nation's economy has \ldots{}?

\begin{verbatim}
                   economy_retro      n
              Gotten Much Better  15621
 Gotten Better / Somewhat Better  80090
           Stayed About The Same  95382
   Gotten Worse / Somewhat Worse 102542
               Gotten Much Worse  73450
                        Not Sure   6739
                            <NA>    732
\end{verbatim}

\begin{itemize}
\tightlist
\item
  Years: All of 2006-2016
\item
  Limitations: Response options varies by year. Some are collapsed into
  one category (e.g. \texttt{Gotten\ Better}, presented in some years,
  and \texttt{Gotten\ Somewhat\ Better}, presented in other years, are
  collapsed into \texttt{Gotten\ Better\ /\ Somewhat\ Better}). Some are
  left as is. For example, \texttt{Not\ Sure} was not an option in 2009.
\end{itemize}

\subsection{Approval}\label{approval}

\subsubsection{\texorpdfstring{\texttt{approval\_pres}: President
approval}{approval\_pres: President approval}}\label{approval_pres-president-approval}

Do you approve of the way each is doing their job\ldots{} {[}Pipe
Incumbent President{]}

\begin{verbatim}
                  approval_pres      n
               Strongly Approve  74158
               Somewhat Approve  92205
            Somewhat Disapprove  40113
            Strongly Disapprove 156820
                       Not Sure  10117
 Neither Approve Nor Disapprove    443
                           <NA>    700
\end{verbatim}

\begin{itemize}
\tightlist
\item
  Years: All of 2006-2016
\item
  Limitations: \texttt{Neither\ approve\ nor\ disapprove} only included
  in 2007.
\item
  This question is asked in a grid format, along with Governors,
  Congress, and Courts.
\end{itemize}

\subsubsection{\texorpdfstring{\texttt{approval\_rep}: House
Representative
approval}{approval\_rep: House Representative approval}}\label{approval_rep-house-representative-approval}

Do you approve of the way each is doing their job\ldots{} {[}Pipe
Incumbent Representative's Name{]}

\begin{verbatim}
                     approval_rep      n
                 Strongly Approve  55069
       Approve / Somewhat Approve 116457
 Disapprove / Somewhat Disapprove  65067
              Strongly Disapprove  58680
           Never Heard / Not Sure  71268
   Neither Approve Nor Disapprove   1798
                             <NA>   6217
\end{verbatim}

\begin{itemize}
\tightlist
\item
  Years: All of 2006-2016
\item
  Limitations: \texttt{Neither\ approve\ nor\ disapprove} only included
  in 2007.
\item
  This question is asked in a grid format, along with Senators
  (\texttt{approval\_sen1}, \texttt{approval\_sen2}).
\item
  To see who {[}Representative{]} refers to for a particular respondent,
  see \texttt{rep\_inc} (incumbent identifier in \texttt{rep\_icpsr})
\end{itemize}

\subsubsection{\texorpdfstring{\texttt{approval\_sen1}: Senator 1
approval}{approval\_sen1: Senator 1 approval}}\label{approval_sen1-senator-1-approval}

Do you approve of the way each is doing their job\ldots{} {[}Pipe
Incumbent Senator 1's Name{]}

\begin{verbatim}
                    approval_sen1      n
                 Strongly Approve  49354
       Approve / Somewhat Approve 118550
 Disapprove / Somewhat Disapprove  74192
              Strongly Disapprove  72873
           Never Heard / Not Sure  53904
   Neither Approve Nor Disapprove   1414
                             <NA>   4269
\end{verbatim}

\begin{itemize}
\tightlist
\item
  Years: All of 2006-2016
\item
  Limitations: : Response options varies by year. Some are collapsed
  into one category (e.g. \texttt{Approve}, presented in some years, and
  \texttt{Somewhat\ Approve}, presented in other years, are collapsed
  into \texttt{Approve\ /\ Somewhat\ Approve}).
  \texttt{Neither\ approve\ nor\ disapprove} only included in 2007.
\item
  To see who {[}Senator 1{]} refers to for a particular respondent, see
  \texttt{sen1\_inc} (incumbent identifier in \texttt{sen1\_icpsr})
\end{itemize}

\subsubsection{\texorpdfstring{\texttt{approval\_sen2}: Senator 2
approval}{approval\_sen2: Senator 2 approval}}\label{approval_sen2-senator-2-approval}

Do you approve of the way each is doing their job\ldots{} {[}Pipe
Incumbent Senator 2's Name{]}

\begin{verbatim}
                    approval_sen2      n
                 Strongly Approve  51255
       Approve / Somewhat Approve 114286
 Disapprove / Somewhat Disapprove  73083
              Strongly Disapprove  73362
           Never Heard / Not Sure  56368
   Neither Approve Nor Disapprove   1158
                             <NA>   5044
\end{verbatim}

\begin{itemize}
\tightlist
\item
  See \texttt{approval\_sen2}
\end{itemize}

\subsubsection{\texorpdfstring{\texttt{approval\_gov}: Governor
approval}{approval\_gov: Governor approval}}\label{approval_gov-governor-approval}

Do you approve of the way each is doing their job\ldots{} Governor of
{[}Pipe State{]}

\begin{verbatim}
                   approval_gov      n
               Strongly Approve  54544
               Somewhat Approve 116578
            Somewhat Disapprove  71062
            Strongly Disapprove  97473
                       Not Sure  31575
 Neither Approve Nor Disapprove   1414
                           <NA>   1910
\end{verbatim}

\begin{itemize}
\tightlist
\item
  Years: All of 2006-2016
\item
  Limitations: See \texttt{approval\_pres}
\item
  To see who the Governor refers to for a particular respondent, see
  \texttt{gov\_inc} (incumbent identifier in \texttt{gov\_fec}, if
  applicable).
\end{itemize}

\newpage

\section{Presidential Vote}\label{presidential-vote}

\begin{tcolorbox}[title={A note on \texttt{intent} and \texttt{voted}}]
In this dataset we make the distinction between "intent" / "preference" vs. "voted" / "vote choice". "Intent" (or "preference") refers to the response to the prospective question of the sort "who would you vote for?" in the \emph{pre-election} wave. "Vote choice" refers to the response to the retrospective question of the sort "in the election this November, who did you vote for?" Response to the vote choice questions coalesces both \emph{post-election} wave responses (the bulk of the responses) and pre-election respondents who reported having already voted early. 
\end{tcolorbox}

\subsubsection{\texorpdfstring{\texttt{intent\_pres\_08}: 2008 President
preference (before
voting)}{intent\_pres\_08: 2008 President preference (before voting)}}\label{intent_pres_08-2008-president-preference-before-voting}

For which candidate for President of the United States would you vote?

\begin{verbatim}
                intent_pres_08      n
                   John McCain  13322
                  Barack Obama  12897
                      Ron Paul    535
                   Ralph Nader    209
                      Bob Barr    258
              Cynthia McKinney     74
                         Other    352
 I Won't Vote In This Election    851
                  I'm Not Sure   1697
                          <NA> 344361
\end{verbatim}

\begin{itemize}
\tightlist
\item
  Years: 2008
\end{itemize}

\subsubsection{\texorpdfstring{\texttt{intent\_pres\_12}: 2012 President
preference (before
voting)}{intent\_pres\_12: 2012 President preference (before voting)}}\label{intent_pres_12-2012-president-preference-before-voting}

In the race for President of the United States, who do you prefer?

\begin{verbatim}
                   intent_pres_12      n
         Mitt Romney (Republican)  20738
        Barack Obama (Democratic)  24401
                            Other   1781
 I Will Not Vote In This Election   1467
                     I'm Not Sure   3856
                             <NA> 322313
\end{verbatim}

\begin{itemize}
\tightlist
\item
  Years: 2012
\end{itemize}

\subsubsection{\texorpdfstring{\texttt{intent\_pres\_16}: 2016 President
preference (before
voting)}{intent\_pres\_16: 2016 President preference (before voting)}}\label{intent_pres_16-2016-president-preference-before-voting}

Which candidate did you prefer for President of the United States?

\begin{verbatim}
                intent_pres_16      n
     Donald Trump (Republican)  19227
    Hillary Clinton (Democrat)  27502
    Gary Johnson (Libertarian)   3145
            Jill Stein (Green)   1400
                         Other   1880
 I Won't Vote In This Election   3312
                  I'm Not Sure   6536
                          <NA> 311554
\end{verbatim}

\begin{itemize}
\tightlist
\item
  Years: 2016
\end{itemize}

\subsubsection{\texorpdfstring{\texttt{voted\_pres\_08}: 2008 President
vote choice (after
voting)}{voted\_pres\_08: 2008 President vote choice (after voting)}}\label{voted_pres_08-2008-president-vote-choice-after-voting}

2008: For which candidate for President of the United States did you
vote? {[}see guide for wording in all years{]}

\begin{verbatim}
             voted_pres_08      n
 Barack Obama (Democratic)  73986
  John McCain (Republican)  68398
              Someone Else   4204
              Did Not Vote  18227
              Don't Recall   1787
                      <NA> 207954
\end{verbatim}

\begin{itemize}
\tightlist
\item
  Years: 2008, 2009, 2010, 2011, 2012
\item
  Limitations: Response options offer slightly by year; some are
  collapsed into one.
\end{itemize}

\subsubsection{\texorpdfstring{\texttt{voted\_pres\_12}: 2012 President
vote choice (after
voting)}{voted\_pres\_12: 2012 President vote choice (after voting)}}\label{voted_pres_12-2012-president-vote-choice-after-voting}

2012: For whom did you vote for President of the United States? 2016: In
2012, who did you vote for in the election for President? {[}see guide
for wording in all years{]}

\begin{verbatim}
               voted_pres_12      n
                Barack Obama  82681
                 Mitt Romney  64956
        Other / Someone Else   5890
                Did Not Vote   2758
     Not Sure / Don't Recall   1990
 I Did Not Vote In This Race     81
                        <NA> 216200
\end{verbatim}

\begin{itemize}
\tightlist
\item
  Years: 2012, 2013, 2014, 2015, 2016
\item
  Limitations: Response options offer slightly by year; some are
  collapsed into one.
\item
  This variable coalesces two variables: Either the response to the
  early vote question in the pre-election wave if the respondent
  indicates they have already voted, or if not, the response in the
  post-election wave.
\end{itemize}

\subsubsection{\texorpdfstring{\texttt{voted\_pres\_16}: 2016 President
vote choice (after
voting)}{voted\_pres\_16: 2016 President vote choice (after voting)}}\label{voted_pres_16-2016-president-vote-choice-after-voting}

For whom did you vote for President of the United States?
{[}Post-election{]}

\begin{verbatim}
                  voted_pres_16      n
      Donald Trump (Republican)  18836
     Hillary Clinton (Democrat)  22284
     Gary Johnson (Libertarian)   1865
             Jill Stein (Green)    926
                          Other   1147
 I Didn't Vote In This Election     91
                   I'm Not Sure    240
    Evan McMullin (Independent)    163
                           <NA> 329004
\end{verbatim}

\begin{itemize}
\tightlist
\item
  Years: 2016
\item
  This variable coalesces two variables in the CCES: Either the response
  to the early vote question in the pre-election wave if the respondent
  indicates they have already voted, or if not, the response in the
  post-election wave.
\end{itemize}

\section{House, Senate and Governor
Voting}\label{house-senate-and-governor-voting}

\subsection{Preference}\label{preference}

\subsubsection{\texorpdfstring{\texttt{intent\_rep}: House preference
(before
voting)}{intent\_rep: House preference (before voting)}}\label{intent_rep-house-preference-before-voting}

In the general election for U.S. House of Representatives in your area,
who do you prefer?

\begin{verbatim}
                           intent_rep      n
             [Democrat / Candidate 1] 103873
           [Republican / Candidate 2]  97039
                [Other / Candidate 3]   4071
   $HouseCand4Name ($HouseCand4Party)     18
                                Other   1720
                         I'm Not Sure  60579
                               No One  15860
   $HouseCand5Name ($HouseCand5Party)     20
        I Won't Vote In This Election   2269
   $HouseCand6Name ($HouseCand6Party)     19
   $HouseCand7Name ($HouseCand7Party)     15
   $HouseCand8Name ($HouseCand8Party)     14
   $HouseCand9Name ($HouseCand9Party)      1
 $HouseCand10Name ($HouseCand10Party)      1
 $HouseCand11Name ($HouseCand11Party)      3
                                 <NA>  89054
\end{verbatim}

\begin{itemize}
\tightlist
\item
  Years: 2006, 2008, 2010, 2012, 2014, 2016
\item
  Limitations: Only available for even years. The third party candidate
  not specified for early years. The fourth candidate and onwards not
  shown for most years. Response options differ by year.
\item
  Note that for each respondent, a name (and party affiliation) is shown
  in place of the square bracket values. To see the candidate chosen,
  see \texttt{intent\_rep\_chosen}.
  \texttt{{[}Other\ /\ Candidate\ 3{]}} refers to the third option
  presented, whereas \texttt{Other} refers to the unnamed choice after
  all numbered candidates.
\end{itemize}

\subsubsection{\texorpdfstring{\texttt{intent\_sen}: Senate preference
(before
voting)}{intent\_sen: Senate preference (before voting)}}\label{intent_sen-senate-preference-before-voting}

In the race for U.S. Senator in your state, who do you prefer?

\begin{verbatim}
                     intent_sen      n
       [Democrat / Candidate 1]  78318
     [Republican / Candidate 2]  68733
          [Other / Candidate 3]   4113
 $SenCand4Name ($SenCand4Party)     19
                          Other   1188
                   I'm Not Sure  31681
                         No One   9493
  I Won't Vote In This Election   1145
                           <NA> 179866
\end{verbatim}

\begin{itemize}
\tightlist
\item
  Years: 2006, 2008, 2010, 2012, 2014, 2016
\item
  Limitations: See \texttt{intente\_rep}. When both senate seats are up
  for re-election in the same year, only responses to the first senate
  seat is incorporated. For the second senate seat, see individual
  year's CCES.
\end{itemize}

\subsubsection{\texorpdfstring{\texttt{intent\_gov}: Governor preference
(before
voting)}{intent\_gov: Governor preference (before voting)}}\label{intent_gov-governor-preference-before-voting}

In the race for Governor in your state, who do you prefer?

\begin{verbatim}
                    intent_gov      n
      [Democrat / Candidate 1]  55600
    [Republican / Candidate 2]  50244
         [Other / Candidate 3]   3681
                         Other    882
                  I'm Not Sure  18342
                        No One   5723
 I Won't Vote In This Election    466
                          <NA> 239618
\end{verbatim}

\begin{itemize}
\tightlist
\item
  Years: 2006, 2008, 2010, 2012, 2014, 2016
\item
  Limitations: See \texttt{intente\_rep}. For governor elections in odd
  years, see individual year's CCES.
\end{itemize}

\subsection{Vote Choice}\label{vote-choice}

\subsubsection{\texorpdfstring{\texttt{voted\_rep}: House vote choice
(after
voting)}{voted\_rep: House vote choice (after voting)}}\label{voted_rep-house-vote-choice-after-voting}

For whom did you vote for U.S. House?

\begin{verbatim}
                            voted_rep      n
             [Democrat / Candidate 1]  94662
           [Republican / Candidate 2]  94122
                [Other / Candidate 3]   2571
   $HouseCand4Name ($HouseCand4Party)     15
                                Other   2434
          I Did Not Vote In This Race  11591
   $HouseCand5Name ($HouseCand5Party)     22
                             Not Sure   4020
   $HouseCand6Name ($HouseCand6Party)     15
   $HouseCand7Name ($HouseCand7Party)     13
   $HouseCand8Name ($HouseCand8Party)     16
   $HouseCand9Name ($HouseCand9Party)      2
 $HouseCand10Name ($HouseCand10Party)      2
 $HouseCand11Name ($HouseCand11Party)      3
                                 <NA> 165068
\end{verbatim}

\begin{itemize}
\tightlist
\item
  Years: 2006, 2008, 2010, 2012, 2014, 2016
\item
  This variable coalesces two variables in the CCES for years 2012 and
  onwards: Either the response to the early vote question in the
  pre-election wave if the respondent indicates they have already voted,
  or if not, the response in the post-election wave.
\end{itemize}

\subsubsection{\texorpdfstring{\texttt{voted\_sen}: Senate vote choice
(after
voting)}{voted\_sen: Senate vote choice (after voting)}}\label{voted_sen-senate-vote-choice-after-voting}

For whom did you vote for U.S. Senator?

\begin{verbatim}
                      voted_sen      n
       [Democrat / Candidate 1]  68808
     [Republican / Candidate 2]  63844
          [Other / Candidate 3]   2743
                          Other   1624
                       Not Sure   1849
 $SenCand4Name ($SenCand4Party)     11
    I Did Not Vote In This Race   4108
                           <NA> 231569
\end{verbatim}

\begin{itemize}
\tightlist
\item
  Years: 2006, 2008, 2010, 2012, 2014, 2016
\item
  This variable coalesces two variables in the CCES for years 2012 and
  onwards: Either the response to the early vote question in the
  pre-election wave if the respondent indicates they have already voted,
  or if not, the response in the post-election wave.
\end{itemize}

\subsubsection{\texorpdfstring{\texttt{voted\_gov}: Governor vote choice
(after
voting)}{voted\_gov: Governor vote choice (after voting)}}\label{voted_gov-governor-vote-choice-after-voting}

For whom did you vote for Governor?

\begin{verbatim}
                   voted_gov      n
    [Democrat / Candidate 1]  46504
  [Republican / Candidate 2]  45056
       [Other / Candidate 3]   2466
                       Other   1162
 I Did Not Vote In This Race   4509
                    Not Sure    911
                        <NA> 273948
\end{verbatim}

\begin{itemize}
\tightlist
\item
  Years: 2006, 2008, 2010, 2012, 2014, 2016
\item
  This variable coalesces two variables in the CCES for years 2012 and
  onwards: Either the response to the early vote question in the
  pre-election wave if the respondent indicates they have already voted,
  or if not, the response in the post-election wave.
\end{itemize}

\newpage

\section{Text}\label{text}

\subsection{Identifiers}\label{identifiers}

The case identifier \texttt{case\_id} is unique within the year and is
identical to the case identifiers in the individual year's CCES. It
should be used in conjunction with \texttt{year} for a unique identifier
for the whole dataset. Some individuals across years may be the same
YouGov panel respondent with different identifiers; for example the 2007
CCES draws from the 2006 CCES respondents.

\begin{verbatim}
Observations: 374,556
Variables: 2
$ year    <int> 2006, 2006, 2006, 2006, 2006, 2006, 2006, 2006, 2006, ...
$ case_id <int> 439219, 439224, 439228, 439237, 439238, 439242, 439251...
\end{verbatim}

\subsection{Current Representatives}\label{current-representatives}

\subsubsection{Name and Party}\label{name-and-party}

The four names in the three offices that represent the respondent
\emph{at the time of the survey}. Parties are not shown if the
particular year's CCES did not show party. Party names are also
abbreviated down to initials (\texttt{D} for Democrat, \texttt{R} for
Republican, \texttt{I} for Independent) in this dataset.

\begin{verbatim}
Observations: 374,556
Variables: 4
$ rep_current  <chr> "Patrick T. McHenry (R)", "Michael R. Turner (R)"...
$ sen1_current <chr> "Elizabeth Dole (R)", "Mike DeWine (R)", "Robert ...
$ sen2_current <chr> "Richard Burr (R)", "George V. Voinovich (R)", "F...
$ gov_current  <chr> "Michael Easley (D)", "Bob Taft (R)", "Jon Corzin...
\end{verbatim}

\subsubsection{Incumbent Identifiers}\label{incumbent-identifiers}

Unique identifiers (ICPSR / Nominate for Congress, FEC for Governor) for
the current representatives. Identifiers are not part of the individual
year's CCES but merged on for this cumulative dataset only.

The matching of identifiers to respondent occurs through matching by
district, by district and last name, or both:

\begin{itemize}
\tightlist
\item
  For House representatives, we join on \texttt{cong}, \texttt{st}, and
  \texttt{dist} to a NOMINATE database that only consists of unique
  observations according to the key. For duplicates with regards to
  these three variables (e.g.~in the rare case where a new
  representative comes into office mid-session), we match on
  \texttt{cong}, \texttt{st}, \texttt{dist} and last name.
\item
  For Senators, we join entirely on \texttt{cong}, \texttt{st}, and last
  name
\item
  For Governors, we join only on \texttt{st} and last name. In this
  period, there are no two governors in the same state that share the
  same last name.
\end{itemize}

\begin{verbatim}
Observations: 374,556
Variables: 4
$ rep_icpsr  <int> 20522, 20342, 29132, 29911, 29380, 20531, 29126, 29...
$ sen1_icpsr <int> 40303, 15020, 29373, 15021, 14858, 49306, 40101, 15...
$ sen2_icpsr <int> 29548, 49903, 14914, 40502, 40105, 40305, 40302, 29...
$ gov_fec    <chr> "NC5998", NA, "NJ6395", "IL7", NA, "TX3156", "MN472...
\end{verbatim}

\begin{itemize}
\tightlist
\item
  Years: All of 2006-2016
\item
  Limitations: Matching procedure may be incomplete or inaccurate.
\end{itemize}

The unique identifiers can be used to join with other databases to
append additional information such as committee membership and ideology
scores, such as

\begin{quote}
Lewis, Jeffrey B., Keith Poole, Howard Rosenthal, Adam Boche, Aaron
Rudkin, and Luke Sonnet (2017). Voteview: Congressional Roll-Call Votes
Database. \url{https://voteview.com/}
\end{quote}

\subsection{Candidates}\label{candidates}

The text responses that the respondent chose in each of the
\texttt{intent\_} / \texttt{voted\_} questions, if the respondent was a
candidate. For example, respondent with \texttt{case\_id\ =\ 163051575}
in the 2012 CCES chose the first option in the House representative
preference question. \texttt{intent\_rep\_chosen} shows that for this
particular respondent, the first option was
\texttt{Maxine\ Waters\ (Democrat)} who has a FEC Identifier of
\texttt{H4CA23011}.

\begin{Shaded}
\begin{Highlighting}[]
\NormalTok{df }\OperatorTok\StringTok{ }
\StringTok{  }\KeywordTok{filter}\NormalTok{(year }\OperatorTok{==}\StringTok{ }\DecValTok{2012}\NormalTok{, st }\OperatorTok{==}\StringTok{ "CA"}\NormalTok{, dist_up }\OperatorTok{==}\StringTok{ }\DecValTok{43}\NormalTok{) }\OperatorTok\StringTok{ }
\StringTok{  }\KeywordTok{select}\NormalTok{(}\KeywordTok{matches}\NormalTok{(}\StringTok{"intent_rep"}\NormalTok{)) }
\end{Highlighting}
\end{Shaded}

\begin{verbatim}
# A tibble: 91 x 3
   intent_rep                 intent_rep_chosen intent_rep_fec
   <fct>                      <chr>             <chr>         
 1 [Democrat / Candidate 1]   Maxine Waters (D) H4CA23011     
 2 I'm Not Sure               <NA>              <NA>          
 3 No One                     <NA>              <NA>          
 4 [Democrat / Candidate 1]   Maxine Waters (D) H4CA23011     
 5 [Republican / Candidate 2] Bob Flores (D)    H2CA43385     
 6 I'm Not Sure               <NA>              <NA>          
 7 Other                      <NA>              <NA>          
 8 [Republican / Candidate 2] Bob Flores (D)    H2CA43385     
 9 [Republican / Candidate 2] Bob Flores (D)    H2CA43385     
10 [Democrat / Candidate 1]   Maxine Waters (D) H4CA23011     
# ... with 81 more rows
\end{verbatim}

The name and party are those as provided in the CCES datasets (e.g.~in
the form \texttt{HouseCand1Name}). The FEC ID is not part of the CCES
but joined in this dataset.

For all three offices, the matching generally occurs by \texttt{year},
\texttt{st}, \texttt{dist\_up} (not \texttt{dist}, because
\texttt{dist\_up}, refers to the district of the upcoming session) and
\texttt{party}. \texttt{party} is the party affiliation as indicated in
the CCES. For years 2008 and 2010, the first option is automatically
labelled as a Democrat and the second option as a Republican, although
these may be inaccurate at times.

The FEC database originates from

\begin{quote}
Bonica, Adam , 2015, ``Database on Ideology, Money in Politics, and
Elections (DIME)'', \url{doi:10.7910/DVN/O5PX0B}, Harvard Dataverse, V2
\end{quote}

which helpfully includes candidates office sought, district (for House
members), party affiliation, and cycle in which the candidate filed. The
variable \texttt{cycle} in Bonica's data is used to join on the CCES
dataset's \texttt{year} variable.

Only candidates who are unique within the district and party are
considered for the first join. However, many candidates are not unique
within the district-party, as many co-partisans may file in the same
district. The second matching process thus considers the full name of
the candidate listed in the CCES and the candidates in the FEC database.
\emph{Within} the subset of year, district, and party, a Jaro-Winker
string distance (that ranges from 0 to 1) is computed for both last name
and the first name - middle name. If the sum of the two string distances
are more than 0.2 for all possible combinations, no match is returned.
If there is a unique combination that achieves a unique minimum that is
below 0.2, that combination is declared a match. If there are multiple
matches with the same minimum string distance, one is randomly chosen.

\subsubsection{Chosen}\label{chosen}

\begin{verbatim}
Observations: 374,556
Variables: 6
$ intent_rep_chosen <chr> "Richard C. Carsner (D)", "Stephanie Studeba...
$ intent_sen_chosen <chr> NA, "Sherrod C. Brown (D)", "Robert Menendez...
$ intent_gov_chosen <chr> NA, "Ted Strickland (D)", NA, "Rod Blagojevi...
$ voted_rep_chosen  <chr> "Richard C. Carsner (D)", "Stephanie Studeba...
$ voted_sen_chosen  <chr> NA, "Sherrod C. Brown (D)", "Robert Menendez...
$ voted_gov_chosen  <chr> NA, "Ted Strickland (D)", NA, "Rod Blagojevi...
\end{verbatim}

\begin{itemize}
\tightlist
\item
  Years: 2006, 2008, 2010, 2012, 2014, 2016
\item
  Early years may mislabel the candidate's party, especially when the
  two candidates are of the same party (as in top-two primary states)
\end{itemize}

\subsubsection{Candidate Identifiers}\label{candidate-identifiers}

\begin{verbatim}
Observations: 374,556
Variables: 6
$ intent_rep_fec <chr> "H6NC10141", "H6OH03142", "H0NJ01066", "H8IL090...
$ intent_sen_fec <chr> NA, "S6OH00163", "S6NJ00289", NA, NA, NA, "S6MN...
$ intent_gov_fec <chr> NA, "OH19691", NA, "IL7", "NY19490", NA, "MN472...
$ voted_rep_fec  <chr> "H6NC10141", "H6OH03142", "H0NJ01066", "H8IL090...
$ voted_sen_fec  <chr> NA, "S6OH00163", "S6NJ00289", NA, "S0NY00188", ...
$ voted_gov_fec  <chr> NA, "OH19691", NA, "IL7", "NY19490", NA, "MN472...
\end{verbatim}

\begin{itemize}
\tightlist
\item
  Years: 2006, 2008, 2010, 2012, 2014, 2016
\item
  Limitations: Matching may be inaccurate (see previous section on
  matching methodology). In particular, a lack of a FEC ID may either
  indicate a failure of the matching procedure, or that the candidate in
  question did not file under the FEC. The match rate in the current
  procedure is upwards of 80 percent in the current procedure.
\end{itemize}
\end{document}

